
% Default to the notebook output style

    


% Inherit from the specified cell style.




    
\documentclass[11pt]{article}

    
    
    \usepackage[T1]{fontenc}
    % Nicer default font (+ math font) than Computer Modern for most use cases
    \usepackage{mathpazo}

    % Basic figure setup, for now with no caption control since it's done
    % automatically by Pandoc (which extracts ![](path) syntax from Markdown).
    \usepackage{graphicx}
    % We will generate all images so they have a width \maxwidth. This means
    % that they will get their normal width if they fit onto the page, but
    % are scaled down if they would overflow the margins.
    \makeatletter
    \def\maxwidth{\ifdim\Gin@nat@width>\linewidth\linewidth
    \else\Gin@nat@width\fi}
    \makeatother
    \let\Oldincludegraphics\includegraphics
    % Set max figure width to be 80% of text width, for now hardcoded.
    \renewcommand{\includegraphics}[1]{\Oldincludegraphics[width=.8\maxwidth]{#1}}
    % Ensure that by default, figures have no caption (until we provide a
    % proper Figure object with a Caption API and a way to capture that
    % in the conversion process - todo).
    \usepackage{caption}
    \DeclareCaptionLabelFormat{nolabel}{}
    \captionsetup{labelformat=nolabel}

    \usepackage{adjustbox} % Used to constrain images to a maximum size 
    \usepackage{xcolor} % Allow colors to be defined
    \usepackage{enumerate} % Needed for markdown enumerations to work
    \usepackage{geometry} % Used to adjust the document margins
    \usepackage{amsmath} % Equations
    \usepackage{amssymb} % Equations
    \usepackage{textcomp} % defines textquotesingle
    % Hack from http://tex.stackexchange.com/a/47451/13684:
    \AtBeginDocument{%
        \def\PYZsq{\textquotesingle}% Upright quotes in Pygmentized code
    }
    \usepackage{upquote} % Upright quotes for verbatim code
    \usepackage{eurosym} % defines \euro
    \usepackage[mathletters]{ucs} % Extended unicode (utf-8) support
    \usepackage[utf8x]{inputenc} % Allow utf-8 characters in the tex document
    \usepackage{fancyvrb} % verbatim replacement that allows latex
    \usepackage{grffile} % extends the file name processing of package graphics 
                         % to support a larger range 
    % The hyperref package gives us a pdf with properly built
    % internal navigation ('pdf bookmarks' for the table of contents,
    % internal cross-reference links, web links for URLs, etc.)
    \usepackage{hyperref}
    \usepackage{longtable} % longtable support required by pandoc >1.10
    \usepackage{booktabs}  % table support for pandoc > 1.12.2
    \usepackage[inline]{enumitem} % IRkernel/repr support (it uses the enumerate* environment)
    \usepackage[normalem]{ulem} % ulem is needed to support strikethroughs (\sout)
                                % normalem makes italics be italics, not underlines
    

    
    
    % Colors for the hyperref package
    \definecolor{urlcolor}{rgb}{0,.145,.698}
    \definecolor{linkcolor}{rgb}{.71,0.21,0.01}
    \definecolor{citecolor}{rgb}{.12,.54,.11}

    % ANSI colors
    \definecolor{ansi-black}{HTML}{3E424D}
    \definecolor{ansi-black-intense}{HTML}{282C36}
    \definecolor{ansi-red}{HTML}{E75C58}
    \definecolor{ansi-red-intense}{HTML}{B22B31}
    \definecolor{ansi-green}{HTML}{00A250}
    \definecolor{ansi-green-intense}{HTML}{007427}
    \definecolor{ansi-yellow}{HTML}{DDB62B}
    \definecolor{ansi-yellow-intense}{HTML}{B27D12}
    \definecolor{ansi-blue}{HTML}{208FFB}
    \definecolor{ansi-blue-intense}{HTML}{0065CA}
    \definecolor{ansi-magenta}{HTML}{D160C4}
    \definecolor{ansi-magenta-intense}{HTML}{A03196}
    \definecolor{ansi-cyan}{HTML}{60C6C8}
    \definecolor{ansi-cyan-intense}{HTML}{258F8F}
    \definecolor{ansi-white}{HTML}{C5C1B4}
    \definecolor{ansi-white-intense}{HTML}{A1A6B2}

    % commands and environments needed by pandoc snippets
    % extracted from the output of `pandoc -s`
    \providecommand{\tightlist}{%
      \setlength{\itemsep}{0pt}\setlength{\parskip}{0pt}}
    \DefineVerbatimEnvironment{Highlighting}{Verbatim}{commandchars=\\\{\}}
    % Add ',fontsize=\small' for more characters per line
    \newenvironment{Shaded}{}{}
    \newcommand{\KeywordTok}[1]{\textcolor[rgb]{0.00,0.44,0.13}{\textbf{{#1}}}}
    \newcommand{\DataTypeTok}[1]{\textcolor[rgb]{0.56,0.13,0.00}{{#1}}}
    \newcommand{\DecValTok}[1]{\textcolor[rgb]{0.25,0.63,0.44}{{#1}}}
    \newcommand{\BaseNTok}[1]{\textcolor[rgb]{0.25,0.63,0.44}{{#1}}}
    \newcommand{\FloatTok}[1]{\textcolor[rgb]{0.25,0.63,0.44}{{#1}}}
    \newcommand{\CharTok}[1]{\textcolor[rgb]{0.25,0.44,0.63}{{#1}}}
    \newcommand{\StringTok}[1]{\textcolor[rgb]{0.25,0.44,0.63}{{#1}}}
    \newcommand{\CommentTok}[1]{\textcolor[rgb]{0.38,0.63,0.69}{\textit{{#1}}}}
    \newcommand{\OtherTok}[1]{\textcolor[rgb]{0.00,0.44,0.13}{{#1}}}
    \newcommand{\AlertTok}[1]{\textcolor[rgb]{1.00,0.00,0.00}{\textbf{{#1}}}}
    \newcommand{\FunctionTok}[1]{\textcolor[rgb]{0.02,0.16,0.49}{{#1}}}
    \newcommand{\RegionMarkerTok}[1]{{#1}}
    \newcommand{\ErrorTok}[1]{\textcolor[rgb]{1.00,0.00,0.00}{\textbf{{#1}}}}
    \newcommand{\NormalTok}[1]{{#1}}
    
    % Additional commands for more recent versions of Pandoc
    \newcommand{\ConstantTok}[1]{\textcolor[rgb]{0.53,0.00,0.00}{{#1}}}
    \newcommand{\SpecialCharTok}[1]{\textcolor[rgb]{0.25,0.44,0.63}{{#1}}}
    \newcommand{\VerbatimStringTok}[1]{\textcolor[rgb]{0.25,0.44,0.63}{{#1}}}
    \newcommand{\SpecialStringTok}[1]{\textcolor[rgb]{0.73,0.40,0.53}{{#1}}}
    \newcommand{\ImportTok}[1]{{#1}}
    \newcommand{\DocumentationTok}[1]{\textcolor[rgb]{0.73,0.13,0.13}{\textit{{#1}}}}
    \newcommand{\AnnotationTok}[1]{\textcolor[rgb]{0.38,0.63,0.69}{\textbf{\textit{{#1}}}}}
    \newcommand{\CommentVarTok}[1]{\textcolor[rgb]{0.38,0.63,0.69}{\textbf{\textit{{#1}}}}}
    \newcommand{\VariableTok}[1]{\textcolor[rgb]{0.10,0.09,0.49}{{#1}}}
    \newcommand{\ControlFlowTok}[1]{\textcolor[rgb]{0.00,0.44,0.13}{\textbf{{#1}}}}
    \newcommand{\OperatorTok}[1]{\textcolor[rgb]{0.40,0.40,0.40}{{#1}}}
    \newcommand{\BuiltInTok}[1]{{#1}}
    \newcommand{\ExtensionTok}[1]{{#1}}
    \newcommand{\PreprocessorTok}[1]{\textcolor[rgb]{0.74,0.48,0.00}{{#1}}}
    \newcommand{\AttributeTok}[1]{\textcolor[rgb]{0.49,0.56,0.16}{{#1}}}
    \newcommand{\InformationTok}[1]{\textcolor[rgb]{0.38,0.63,0.69}{\textbf{\textit{{#1}}}}}
    \newcommand{\WarningTok}[1]{\textcolor[rgb]{0.38,0.63,0.69}{\textbf{\textit{{#1}}}}}
    
    
    % Define a nice break command that doesn't care if a line doesn't already
    % exist.
    \def\br{\hspace*{\fill} \\* }
    % Math Jax compatability definitions
    \def\gt{>}
    \def\lt{<}
    % Document parameters
    \title{Pa?ses do Mundo}
    
    
    

    % Pygments definitions
    
\makeatletter
\def\PY@reset{\let\PY@it=\relax \let\PY@bf=\relax%
    \let\PY@ul=\relax \let\PY@tc=\relax%
    \let\PY@bc=\relax \let\PY@ff=\relax}
\def\PY@tok#1{\csname PY@tok@#1\endcsname}
\def\PY@toks#1+{\ifx\relax#1\empty\else%
    \PY@tok{#1}\expandafter\PY@toks\fi}
\def\PY@do#1{\PY@bc{\PY@tc{\PY@ul{%
    \PY@it{\PY@bf{\PY@ff{#1}}}}}}}
\def\PY#1#2{\PY@reset\PY@toks#1+\relax+\PY@do{#2}}

\expandafter\def\csname PY@tok@w\endcsname{\def\PY@tc##1{\textcolor[rgb]{0.73,0.73,0.73}{##1}}}
\expandafter\def\csname PY@tok@c\endcsname{\let\PY@it=\textit\def\PY@tc##1{\textcolor[rgb]{0.25,0.50,0.50}{##1}}}
\expandafter\def\csname PY@tok@cp\endcsname{\def\PY@tc##1{\textcolor[rgb]{0.74,0.48,0.00}{##1}}}
\expandafter\def\csname PY@tok@k\endcsname{\let\PY@bf=\textbf\def\PY@tc##1{\textcolor[rgb]{0.00,0.50,0.00}{##1}}}
\expandafter\def\csname PY@tok@kp\endcsname{\def\PY@tc##1{\textcolor[rgb]{0.00,0.50,0.00}{##1}}}
\expandafter\def\csname PY@tok@kt\endcsname{\def\PY@tc##1{\textcolor[rgb]{0.69,0.00,0.25}{##1}}}
\expandafter\def\csname PY@tok@o\endcsname{\def\PY@tc##1{\textcolor[rgb]{0.40,0.40,0.40}{##1}}}
\expandafter\def\csname PY@tok@ow\endcsname{\let\PY@bf=\textbf\def\PY@tc##1{\textcolor[rgb]{0.67,0.13,1.00}{##1}}}
\expandafter\def\csname PY@tok@nb\endcsname{\def\PY@tc##1{\textcolor[rgb]{0.00,0.50,0.00}{##1}}}
\expandafter\def\csname PY@tok@nf\endcsname{\def\PY@tc##1{\textcolor[rgb]{0.00,0.00,1.00}{##1}}}
\expandafter\def\csname PY@tok@nc\endcsname{\let\PY@bf=\textbf\def\PY@tc##1{\textcolor[rgb]{0.00,0.00,1.00}{##1}}}
\expandafter\def\csname PY@tok@nn\endcsname{\let\PY@bf=\textbf\def\PY@tc##1{\textcolor[rgb]{0.00,0.00,1.00}{##1}}}
\expandafter\def\csname PY@tok@ne\endcsname{\let\PY@bf=\textbf\def\PY@tc##1{\textcolor[rgb]{0.82,0.25,0.23}{##1}}}
\expandafter\def\csname PY@tok@nv\endcsname{\def\PY@tc##1{\textcolor[rgb]{0.10,0.09,0.49}{##1}}}
\expandafter\def\csname PY@tok@no\endcsname{\def\PY@tc##1{\textcolor[rgb]{0.53,0.00,0.00}{##1}}}
\expandafter\def\csname PY@tok@nl\endcsname{\def\PY@tc##1{\textcolor[rgb]{0.63,0.63,0.00}{##1}}}
\expandafter\def\csname PY@tok@ni\endcsname{\let\PY@bf=\textbf\def\PY@tc##1{\textcolor[rgb]{0.60,0.60,0.60}{##1}}}
\expandafter\def\csname PY@tok@na\endcsname{\def\PY@tc##1{\textcolor[rgb]{0.49,0.56,0.16}{##1}}}
\expandafter\def\csname PY@tok@nt\endcsname{\let\PY@bf=\textbf\def\PY@tc##1{\textcolor[rgb]{0.00,0.50,0.00}{##1}}}
\expandafter\def\csname PY@tok@nd\endcsname{\def\PY@tc##1{\textcolor[rgb]{0.67,0.13,1.00}{##1}}}
\expandafter\def\csname PY@tok@s\endcsname{\def\PY@tc##1{\textcolor[rgb]{0.73,0.13,0.13}{##1}}}
\expandafter\def\csname PY@tok@sd\endcsname{\let\PY@it=\textit\def\PY@tc##1{\textcolor[rgb]{0.73,0.13,0.13}{##1}}}
\expandafter\def\csname PY@tok@si\endcsname{\let\PY@bf=\textbf\def\PY@tc##1{\textcolor[rgb]{0.73,0.40,0.53}{##1}}}
\expandafter\def\csname PY@tok@se\endcsname{\let\PY@bf=\textbf\def\PY@tc##1{\textcolor[rgb]{0.73,0.40,0.13}{##1}}}
\expandafter\def\csname PY@tok@sr\endcsname{\def\PY@tc##1{\textcolor[rgb]{0.73,0.40,0.53}{##1}}}
\expandafter\def\csname PY@tok@ss\endcsname{\def\PY@tc##1{\textcolor[rgb]{0.10,0.09,0.49}{##1}}}
\expandafter\def\csname PY@tok@sx\endcsname{\def\PY@tc##1{\textcolor[rgb]{0.00,0.50,0.00}{##1}}}
\expandafter\def\csname PY@tok@m\endcsname{\def\PY@tc##1{\textcolor[rgb]{0.40,0.40,0.40}{##1}}}
\expandafter\def\csname PY@tok@gh\endcsname{\let\PY@bf=\textbf\def\PY@tc##1{\textcolor[rgb]{0.00,0.00,0.50}{##1}}}
\expandafter\def\csname PY@tok@gu\endcsname{\let\PY@bf=\textbf\def\PY@tc##1{\textcolor[rgb]{0.50,0.00,0.50}{##1}}}
\expandafter\def\csname PY@tok@gd\endcsname{\def\PY@tc##1{\textcolor[rgb]{0.63,0.00,0.00}{##1}}}
\expandafter\def\csname PY@tok@gi\endcsname{\def\PY@tc##1{\textcolor[rgb]{0.00,0.63,0.00}{##1}}}
\expandafter\def\csname PY@tok@gr\endcsname{\def\PY@tc##1{\textcolor[rgb]{1.00,0.00,0.00}{##1}}}
\expandafter\def\csname PY@tok@ge\endcsname{\let\PY@it=\textit}
\expandafter\def\csname PY@tok@gs\endcsname{\let\PY@bf=\textbf}
\expandafter\def\csname PY@tok@gp\endcsname{\let\PY@bf=\textbf\def\PY@tc##1{\textcolor[rgb]{0.00,0.00,0.50}{##1}}}
\expandafter\def\csname PY@tok@go\endcsname{\def\PY@tc##1{\textcolor[rgb]{0.53,0.53,0.53}{##1}}}
\expandafter\def\csname PY@tok@gt\endcsname{\def\PY@tc##1{\textcolor[rgb]{0.00,0.27,0.87}{##1}}}
\expandafter\def\csname PY@tok@err\endcsname{\def\PY@bc##1{\setlength{\fboxsep}{0pt}\fcolorbox[rgb]{1.00,0.00,0.00}{1,1,1}{\strut ##1}}}
\expandafter\def\csname PY@tok@kc\endcsname{\let\PY@bf=\textbf\def\PY@tc##1{\textcolor[rgb]{0.00,0.50,0.00}{##1}}}
\expandafter\def\csname PY@tok@kd\endcsname{\let\PY@bf=\textbf\def\PY@tc##1{\textcolor[rgb]{0.00,0.50,0.00}{##1}}}
\expandafter\def\csname PY@tok@kn\endcsname{\let\PY@bf=\textbf\def\PY@tc##1{\textcolor[rgb]{0.00,0.50,0.00}{##1}}}
\expandafter\def\csname PY@tok@kr\endcsname{\let\PY@bf=\textbf\def\PY@tc##1{\textcolor[rgb]{0.00,0.50,0.00}{##1}}}
\expandafter\def\csname PY@tok@bp\endcsname{\def\PY@tc##1{\textcolor[rgb]{0.00,0.50,0.00}{##1}}}
\expandafter\def\csname PY@tok@fm\endcsname{\def\PY@tc##1{\textcolor[rgb]{0.00,0.00,1.00}{##1}}}
\expandafter\def\csname PY@tok@vc\endcsname{\def\PY@tc##1{\textcolor[rgb]{0.10,0.09,0.49}{##1}}}
\expandafter\def\csname PY@tok@vg\endcsname{\def\PY@tc##1{\textcolor[rgb]{0.10,0.09,0.49}{##1}}}
\expandafter\def\csname PY@tok@vi\endcsname{\def\PY@tc##1{\textcolor[rgb]{0.10,0.09,0.49}{##1}}}
\expandafter\def\csname PY@tok@vm\endcsname{\def\PY@tc##1{\textcolor[rgb]{0.10,0.09,0.49}{##1}}}
\expandafter\def\csname PY@tok@sa\endcsname{\def\PY@tc##1{\textcolor[rgb]{0.73,0.13,0.13}{##1}}}
\expandafter\def\csname PY@tok@sb\endcsname{\def\PY@tc##1{\textcolor[rgb]{0.73,0.13,0.13}{##1}}}
\expandafter\def\csname PY@tok@sc\endcsname{\def\PY@tc##1{\textcolor[rgb]{0.73,0.13,0.13}{##1}}}
\expandafter\def\csname PY@tok@dl\endcsname{\def\PY@tc##1{\textcolor[rgb]{0.73,0.13,0.13}{##1}}}
\expandafter\def\csname PY@tok@s2\endcsname{\def\PY@tc##1{\textcolor[rgb]{0.73,0.13,0.13}{##1}}}
\expandafter\def\csname PY@tok@sh\endcsname{\def\PY@tc##1{\textcolor[rgb]{0.73,0.13,0.13}{##1}}}
\expandafter\def\csname PY@tok@s1\endcsname{\def\PY@tc##1{\textcolor[rgb]{0.73,0.13,0.13}{##1}}}
\expandafter\def\csname PY@tok@mb\endcsname{\def\PY@tc##1{\textcolor[rgb]{0.40,0.40,0.40}{##1}}}
\expandafter\def\csname PY@tok@mf\endcsname{\def\PY@tc##1{\textcolor[rgb]{0.40,0.40,0.40}{##1}}}
\expandafter\def\csname PY@tok@mh\endcsname{\def\PY@tc##1{\textcolor[rgb]{0.40,0.40,0.40}{##1}}}
\expandafter\def\csname PY@tok@mi\endcsname{\def\PY@tc##1{\textcolor[rgb]{0.40,0.40,0.40}{##1}}}
\expandafter\def\csname PY@tok@il\endcsname{\def\PY@tc##1{\textcolor[rgb]{0.40,0.40,0.40}{##1}}}
\expandafter\def\csname PY@tok@mo\endcsname{\def\PY@tc##1{\textcolor[rgb]{0.40,0.40,0.40}{##1}}}
\expandafter\def\csname PY@tok@ch\endcsname{\let\PY@it=\textit\def\PY@tc##1{\textcolor[rgb]{0.25,0.50,0.50}{##1}}}
\expandafter\def\csname PY@tok@cm\endcsname{\let\PY@it=\textit\def\PY@tc##1{\textcolor[rgb]{0.25,0.50,0.50}{##1}}}
\expandafter\def\csname PY@tok@cpf\endcsname{\let\PY@it=\textit\def\PY@tc##1{\textcolor[rgb]{0.25,0.50,0.50}{##1}}}
\expandafter\def\csname PY@tok@c1\endcsname{\let\PY@it=\textit\def\PY@tc##1{\textcolor[rgb]{0.25,0.50,0.50}{##1}}}
\expandafter\def\csname PY@tok@cs\endcsname{\let\PY@it=\textit\def\PY@tc##1{\textcolor[rgb]{0.25,0.50,0.50}{##1}}}

\def\PYZbs{\char`\\}
\def\PYZus{\char`\_}
\def\PYZob{\char`\{}
\def\PYZcb{\char`\}}
\def\PYZca{\char`\^}
\def\PYZam{\char`\&}
\def\PYZlt{\char`\<}
\def\PYZgt{\char`\>}
\def\PYZsh{\char`\#}
\def\PYZpc{\char`\%}
\def\PYZdl{\char`\$}
\def\PYZhy{\char`\-}
\def\PYZsq{\char`\'}
\def\PYZdq{\char`\"}
\def\PYZti{\char`\~}
% for compatibility with earlier versions
\def\PYZat{@}
\def\PYZlb{[}
\def\PYZrb{]}
\makeatother


    % Exact colors from NB
    \definecolor{incolor}{rgb}{0.0, 0.0, 0.5}
    \definecolor{outcolor}{rgb}{0.545, 0.0, 0.0}



    
    % Prevent overflowing lines due to hard-to-break entities
    \sloppy 
    % Setup hyperref package
    \hypersetup{
      breaklinks=true,  % so long urls are correctly broken across lines
      colorlinks=true,
      urlcolor=urlcolor,
      linkcolor=linkcolor,
      citecolor=citecolor,
      }
    % Slightly bigger margins than the latex defaults
    
    \geometry{verbose,tmargin=1in,bmargin=1in,lmargin=1in,rmargin=1in}
    
    

    \begin{document}
    
    
    \maketitle
    
    

    
    \hypertarget{extraindo-dados-utilizando-a-biblioteca-python-pandas}{%
\section{Extraindo dados utilizando a biblioteca Python
Pandas}\label{extraindo-dados-utilizando-a-biblioteca-python-pandas}}

Na Indústria 4.0, muito se fala em transferir os dados de produção da
planta para a nuvem, armazená-los em bancos de dados e utilizá-los para
obter `insights' e gerar valor. Porém, como se dá essa transformação das
informações em valor? Existem várias formas de extrair informações
úteis. Uma delas é utilizando Python que atualmente é a linguagem
preferida para trabalhar com dados devido às bibliotecas que oferece aos
usuários, simplicidade da linguagem entre outros. Neste tutorial
utilizarei uma base de dados disponível no Kaggle referente aos países
do mundo para exemplificar como é possível extrair facilmente várias
informações.

    As linhas iniciadas em ``In {[} {]}:'' indicam o código que solicita a
informação. As linhas iniciadas em ``Out{[} {]}:'' indicam as respostas
obtidas.

    Importar a biblioteca pandas

    \begin{Verbatim}[commandchars=\\\{\}]
{\color{incolor}In [{\color{incolor}5}]:} \PY{k+kn}{import} \PY{n+nn}{pandas} \PY{k}{as} \PY{n+nn}{pd}
\end{Verbatim}


    Importar os dados da tabela. O comando `pd.read\_csv()' é um comando
para importar arquivos do tipo .csv, que são arquivos onde os dados
recolhidos são armazenados com separação por vírgulas.

    \begin{Verbatim}[commandchars=\\\{\}]
{\color{incolor}In [{\color{incolor}97}]:} \PY{n}{tab} \PY{o}{=} \PY{n}{pd}\PY{o}{.}\PY{n}{read\PYZus{}csv}\PY{p}{(}\PY{l+s+s1}{\PYZsq{}}\PY{l+s+s1}{countries of the world.csv}\PY{l+s+s1}{\PYZsq{}}\PY{p}{)}
\end{Verbatim}


    Dados gerais da tabela. Com o comando `head(10)' é possível exibir os 10
primeiros dados da tabela.

    \begin{Verbatim}[commandchars=\\\{\}]
{\color{incolor}In [{\color{incolor}96}]:} \PY{n}{tab}\PY{o}{.}\PY{n}{head}\PY{p}{(}\PY{l+m+mi}{10}\PY{p}{)}
\end{Verbatim}


\begin{Verbatim}[commandchars=\\\{\}]
{\color{outcolor}Out[{\color{outcolor}96}]:}               Country                               Region  Population  \textbackslash{}
         0        Afghanistan         ASIA (EX. NEAR EAST)             31056997   
         1            Albania   EASTERN EUROPE                          3581655   
         2            Algeria   NORTHERN AFRICA                        32930091   
         3     American Samoa   OCEANIA                                   57794   
         4            Andorra   WESTERN EUROPE                            71201   
         5             Angola   SUB-SAHARAN AFRICA                     12127071   
         6           Anguilla               LATIN AMER. \& CARIB           13477   
         7  Antigua \& Barbuda               LATIN AMER. \& CARIB           69108   
         8          Argentina               LATIN AMER. \& CARIB        39921833   
         9            Armenia                  C.W. OF IND. STATES      2976372   
         
            Area (sq. mi.) Pop. Density (per sq. mi.) Coastline (coast/area ratio)  \textbackslash{}
         0          647500                       48,0                         0,00   
         1           28748                      124,6                         1,26   
         2         2381740                       13,8                         0,04   
         3             199                      290,4                        58,29   
         4             468                      152,1                         0,00   
         5         1246700                        9,7                         0,13   
         6             102                      132,1                        59,80   
         7             443                      156,0                        34,54   
         8         2766890                       14,4                         0,18   
         9           29800                       99,9                         0,00   
         
           Net migration Infant mortality (per 1000 births)  GDP (\$ per capita)  \textbackslash{}
         0         23,06                             163,07               700.0   
         1         -4,93                              21,52              4500.0   
         2         -0,39                                 31              6000.0   
         3        -20,71                               9,27              8000.0   
         4           6,6                               4,05             19000.0   
         5             0                             191,19              1900.0   
         6         10,76                              21,03              8600.0   
         7         -6,15                              19,46             11000.0   
         8          0,61                              15,18             11200.0   
         9         -6,47                              23,28              3500.0   
         
           Literacy (\%) Phones (per 1000) Arable (\%) Crops (\%) Other (\%) Climate  \textbackslash{}
         0         36,0               3,2      12,13      0,22     87,65       1   
         1         86,5              71,2      21,09      4,42     74,49       3   
         2         70,0              78,1       3,22      0,25     96,53       1   
         3         97,0             259,5         10        15        75       2   
         4        100,0             497,2       2,22         0     97,78       3   
         5         42,0               7,8       2,41      0,24     97,35     NaN   
         6         95,0             460,0          0         0       100       2   
         7         89,0             549,9      18,18      4,55     77,27       2   
         8         97,1             220,4      12,31      0,48     87,21       3   
         9         98,6             195,7      17,55       2,3     80,15       4   
         
           Birthrate Deathrate Agriculture Industry Service  
         0      46,6     20,34        0,38     0,24    0,38  
         1     15,11      5,22       0,232    0,188   0,579  
         2     17,14      4,61       0,101      0,6   0,298  
         3     22,46      3,27         NaN      NaN     NaN  
         4      8,71      6,25         NaN      NaN     NaN  
         5     45,11      24,2       0,096    0,658   0,246  
         6     14,17      5,34        0,04     0,18    0,78  
         7     16,93      5,37       0,038     0,22   0,743  
         8     16,73      7,55       0,095    0,358   0,547  
         9     12,07      8,23       0,239    0,343   0,418  
\end{Verbatim}
            
    Informações sobre as colunas e quantidade de dados da tabela podem ser
obtidas por meio do método `info()'. No resultado abaixo é possível
notar que existem colunas sem dados na tabela, ou seja, nem todas as
colunas possuem 227 dados. Isto pode afetar os resultados obtidos caso
não se realize nenhum tratamento na tabela.

    \begin{Verbatim}[commandchars=\\\{\}]
{\color{incolor}In [{\color{incolor}16}]:} \PY{n}{tab}\PY{o}{.}\PY{n}{info}\PY{p}{(}\PY{p}{)}
\end{Verbatim}


    \begin{Verbatim}[commandchars=\\\{\}]
<class 'pandas.core.frame.DataFrame'>
RangeIndex: 227 entries, 0 to 226
Data columns (total 20 columns):
Country                               227 non-null object
Region                                227 non-null object
Population                            227 non-null int64
Area (sq. mi.)                        227 non-null int64
Pop. Density (per sq. mi.)            227 non-null object
Coastline (coast/area ratio)          227 non-null object
Net migration                         224 non-null object
Infant mortality (per 1000 births)    224 non-null object
GDP (\$ per capita)                    226 non-null float64
Literacy (\%)                          209 non-null object
Phones (per 1000)                     223 non-null object
Arable (\%)                            225 non-null object
Crops (\%)                             225 non-null object
Other (\%)                             225 non-null object
Climate                               205 non-null object
Birthrate                             224 non-null object
Deathrate                             223 non-null object
Agriculture                           212 non-null object
Industry                              211 non-null object
Service                               212 non-null object
dtypes: float64(1), int64(2), object(17)
memory usage: 35.5+ KB

    \end{Verbatim}

    Número de regiões diferentes no mundo e quantidade de países em cada uma
delas. Isto é possível obter por meio do método `value\_counts',
conforme abaixo. Este método fornece a quantidade de valores únicos de
acordo com a coluna, por exemplo abaixo foram encontrados 51 países
diferentes cuja região era `SUB-SAHARAN AFRICA'.

    \begin{Verbatim}[commandchars=\\\{\}]
{\color{incolor}In [{\color{incolor}22}]:} \PY{n}{tab}\PY{p}{[}\PY{l+s+s1}{\PYZsq{}}\PY{l+s+s1}{Region}\PY{l+s+s1}{\PYZsq{}}\PY{p}{]}\PY{o}{.}\PY{n}{value\PYZus{}counts}\PY{p}{(}\PY{p}{)}
\end{Verbatim}


\begin{Verbatim}[commandchars=\\\{\}]
{\color{outcolor}Out[{\color{outcolor}22}]:} SUB-SAHARAN AFRICA                     51
         LATIN AMER. \& CARIB                    45
         WESTERN EUROPE                         28
         ASIA (EX. NEAR EAST)                   28
         OCEANIA                                21
         NEAR EAST                              16
         EASTERN EUROPE                         12
         C.W. OF IND. STATES                    12
         NORTHERN AFRICA                         6
         NORTHERN AMERICA                        5
         BALTICS                                 3
         Name: Region, dtype: int64
\end{Verbatim}
            
    Os cinco maiores países em área. Por meio do método `sort\_values()'
aplicado à coluna correspondente à área de cada um dos países. A opção
`ascending=False' organiza os valores em ordem decrescente.

    \begin{Verbatim}[commandchars=\\\{\}]
{\color{incolor}In [{\color{incolor}94}]:} \PY{n}{tab}\PY{p}{[}\PY{l+s+s1}{\PYZsq{}}\PY{l+s+s1}{Area (sq. mi.)}\PY{l+s+s1}{\PYZsq{}}\PY{p}{]}\PY{o}{.}\PY{n}{sort\PYZus{}values}\PY{p}{(}\PY{n}{ascending}\PY{o}{=}\PY{k+kc}{False}\PY{p}{)}\PY{o}{.}\PY{n}{head}\PY{p}{(}\PY{l+m+mi}{5}\PY{p}{)}
\end{Verbatim}


\begin{Verbatim}[commandchars=\\\{\}]
{\color{outcolor}Out[{\color{outcolor}94}]:} 169    17075200
         36      9984670
         214     9631420
         42      9596960
         27      8511965
         Name: Area (sq. mi.), dtype: int64
\end{Verbatim}
            
    Porém neste filtro aparece apenas o índice dos países de acordo com a
área em ordem decrescente. Para melhor visualização, é possível utilizar
o método `sort\_values( )' diretamente na tabela e não apenas na coluna,
colocando nas opções a coluna de interesse para ser filtrada.

    \begin{Verbatim}[commandchars=\\\{\}]
{\color{incolor}In [{\color{incolor}95}]:} \PY{n}{tab}\PY{o}{.}\PY{n}{sort\PYZus{}values}\PY{p}{(}\PY{l+s+s1}{\PYZsq{}}\PY{l+s+s1}{Area (sq. mi.)}\PY{l+s+s1}{\PYZsq{}}\PY{p}{,} \PY{n}{ascending}\PY{o}{=}\PY{k+kc}{False}\PY{p}{)}\PY{o}{.}\PY{n}{head}\PY{p}{(}\PY{l+m+mi}{5}\PY{p}{)}
\end{Verbatim}


\begin{Verbatim}[commandchars=\\\{\}]
{\color{outcolor}Out[{\color{outcolor}95}]:}             Country                               Region  Population  \textbackslash{}
         169         Russia                  C.W. OF IND. STATES    142893540   
         36          Canada   NORTHERN AMERICA                       33098932   
         214  United States   NORTHERN AMERICA                      298444215   
         42           China         ASIA (EX. NEAR EAST)           1313973713   
         27          Brazil               LATIN AMER. \& CARIB       188078227   
         
              Area (sq. mi.) Pop. Density (per sq. mi.) Coastline (coast/area ratio)  \textbackslash{}
         169        17075200                        8,4                         0,22   
         36          9984670                        3,3                         2,02   
         214         9631420                       31,0                         0,21   
         42          9596960                      136,9                         0,15   
         27          8511965                       22,1                         0,09   
         
             Net migration Infant mortality (per 1000 births)  GDP (\$ per capita)  \textbackslash{}
         169          1,02                              15,39              8900.0   
         36           5,96                               4,75             29800.0   
         214          3,41                                6,5             37800.0   
         42           -0,4                              24,18              5000.0   
         27          -0,03                              29,61              7600.0   
         
             Literacy (\%) Phones (per 1000) Arable (\%) Crops (\%) Other (\%) Climate  \textbackslash{}
         169         99,6             280,6       7,33      0,11     92,56     NaN   
         36          97,0             552,2       4,96      0,02     95,02     NaN   
         214         97,0             898,0      19,13      0,22     80,65       3   
         42          90,9             266,7       15,4      1,25     83,35     1,5   
         27          86,4             225,3       6,96       0,9     92,15       2   
         
             Birthrate Deathrate Agriculture Industry Service  
         169      9,95     14,65       0,054    0,371   0,575  
         36      10,78       7,8       0,022    0,294   0,684  
         214     14,14      8,26        0,01    0,204   0,787  
         42      13,25      6,97       0,125    0,473   0,403  
         27      16,56      6,17       0,084      0,4   0,516  
\end{Verbatim}
            
    De forma análoga, os 5 países com maior população.

    \begin{Verbatim}[commandchars=\\\{\}]
{\color{incolor}In [{\color{incolor}98}]:} \PY{n}{tab}\PY{o}{.}\PY{n}{sort\PYZus{}values}\PY{p}{(}\PY{l+s+s1}{\PYZsq{}}\PY{l+s+s1}{Population}\PY{l+s+s1}{\PYZsq{}}\PY{p}{,} \PY{n}{ascending}\PY{o}{=}\PY{k+kc}{False}\PY{p}{)}\PY{o}{.}\PY{n}{head}\PY{p}{(}\PY{l+m+mi}{5}\PY{p}{)}
\end{Verbatim}


\begin{Verbatim}[commandchars=\\\{\}]
{\color{outcolor}Out[{\color{outcolor}98}]:}             Country                               Region  Population  \textbackslash{}
         42           China         ASIA (EX. NEAR EAST)           1313973713   
         94           India         ASIA (EX. NEAR EAST)           1095351995   
         214  United States   NORTHERN AMERICA                      298444215   
         95       Indonesia         ASIA (EX. NEAR EAST)            245452739   
         27          Brazil               LATIN AMER. \& CARIB       188078227   
         
              Area (sq. mi.) Pop. Density (per sq. mi.) Coastline (coast/area ratio)  \textbackslash{}
         42          9596960                      136,9                         0,15   
         94          3287590                      333,2                         0,21   
         214         9631420                       31,0                         0,21   
         95          1919440                      127,9                         2,85   
         27          8511965                       22,1                         0,09   
         
             Net migration Infant mortality (per 1000 births)  GDP (\$ per capita)  \textbackslash{}
         42           -0,4                              24,18              5000.0   
         94          -0,07                              56,29              2900.0   
         214          3,41                                6,5             37800.0   
         95              0                               35,6              3200.0   
         27          -0,03                              29,61              7600.0   
         
             Literacy (\%) Phones (per 1000) Arable (\%) Crops (\%) Other (\%) Climate  \textbackslash{}
         42          90,9             266,7       15,4      1,25     83,35     1,5   
         94          59,5              45,4       54,4      2,74     42,86     2,5   
         214         97,0             898,0      19,13      0,22     80,65       3   
         95          87,9              52,0      11,32      7,23     81,45       2   
         27          86,4             225,3       6,96       0,9     92,15       2   
         
             Birthrate Deathrate Agriculture Industry Service  
         42      13,25      6,97       0,125    0,473   0,403  
         94      22,01      8,18       0,186    0,276   0,538  
         214     14,14      8,26        0,01    0,204   0,787  
         95      20,34      6,25       0,134    0,458   0,408  
         27      16,56      6,17       0,084      0,4   0,516  
\end{Verbatim}
            
    O país mais densamente povoado pode ser encontrado por meio do método
`max()' que retorna o maior valor entre os valores da coluna escolhida.

    \begin{Verbatim}[commandchars=\\\{\}]
{\color{incolor}In [{\color{incolor}102}]:} \PY{n}{tab}\PY{p}{[}\PY{n}{tab}\PY{p}{[}\PY{l+s+s1}{\PYZsq{}}\PY{l+s+s1}{Pop. Density (per sq. mi.)}\PY{l+s+s1}{\PYZsq{}}\PY{p}{]}\PY{o}{==} \PY{n}{tab}\PY{p}{[}\PY{l+s+s1}{\PYZsq{}}\PY{l+s+s1}{Pop. Density (per sq. mi.)}\PY{l+s+s1}{\PYZsq{}}\PY{p}{]}\PY{o}{.}\PY{n}{max}\PY{p}{(}\PY{p}{)}\PY{p}{]}
\end{Verbatim}


\begin{Verbatim}[commandchars=\\\{\}]
{\color{outcolor}Out[{\color{outcolor}102}]:}     Country                Region  Population  Area (sq. mi.)  \textbackslash{}
          9  Armenia   C.W. OF IND. STATES      2976372           29800   
          
            Pop. Density (per sq. mi.) Coastline (coast/area ratio) Net migration  \textbackslash{}
          9                       99,9                         0,00         -6,47   
          
            Infant mortality (per 1000 births)  GDP (\$ per capita) Literacy (\%)  \textbackslash{}
          9                              23,28              3500.0         98,6   
          
            Phones (per 1000) Arable (\%) Crops (\%) Other (\%) Climate Birthrate  \textbackslash{}
          9             195,7      17,55       2,3     80,15       4     12,07   
          
            Deathrate Agriculture Industry Service  
          9      8,23       0,239    0,343   0,418  
\end{Verbatim}
            
    O país menos densamente povoado pode ser encontrado por meio do método
`min()' que retorna o menor valor entre os valores da coluna escolhida.

    \begin{Verbatim}[commandchars=\\\{\}]
{\color{incolor}In [{\color{incolor}103}]:} \PY{n}{tab}\PY{p}{[}\PY{n}{tab}\PY{p}{[}\PY{l+s+s1}{\PYZsq{}}\PY{l+s+s1}{Pop. Density (per sq. mi.)}\PY{l+s+s1}{\PYZsq{}}\PY{p}{]}\PY{o}{==} \PY{n}{tab}\PY{p}{[}\PY{l+s+s1}{\PYZsq{}}\PY{l+s+s1}{Pop. Density (per sq. mi.)}\PY{l+s+s1}{\PYZsq{}}\PY{p}{]}\PY{o}{.}\PY{n}{min}\PY{p}{(}\PY{p}{)}\PY{p}{]}
\end{Verbatim}


\begin{Verbatim}[commandchars=\\\{\}]
{\color{outcolor}Out[{\color{outcolor}103}]:}        Country                               Region  Population  \textbackslash{}
          80  Greenland   NORTHERN AMERICA                          56361   
          
              Area (sq. mi.) Pop. Density (per sq. mi.) Coastline (coast/area ratio)  \textbackslash{}
          80         2166086                        0,0                         2,04   
          
             Net migration Infant mortality (per 1000 births)  GDP (\$ per capita)  \textbackslash{}
          80         -8,37                              15,82             20000.0   
          
             Literacy (\%) Phones (per 1000) Arable (\%) Crops (\%) Other (\%) Climate  \textbackslash{}
          80          NaN             448,9          0         0       100       1   
          
             Birthrate Deathrate Agriculture Industry Service  
          80     15,93      7,84         NaN      NaN     NaN  
\end{Verbatim}
            
    \begin{Verbatim}[commandchars=\\\{\}]
{\color{incolor}In [{\color{incolor}106}]:} \PY{n}{tab}\PY{p}{[}\PY{n}{tab}\PY{p}{[}\PY{l+s+s1}{\PYZsq{}}\PY{l+s+s1}{Phones (per 1000)}\PY{l+s+s1}{\PYZsq{}}\PY{p}{]}\PY{o}{.}\PY{n}{isnull}\PY{p}{(}\PY{p}{)}\PY{p}{]}
\end{Verbatim}


\begin{Verbatim}[commandchars=\\\{\}]
{\color{outcolor}Out[{\color{outcolor}106}]:}              Country                               Region  Population  \textbackslash{}
          52           Cyprus   NEAR EAST                                784301   
          58       East Timor         ASIA (EX. NEAR EAST)              1062777   
          140      Montserrat               LATIN AMER. \& CARIB            9439   
          223  Western Sahara   NORTHERN AFRICA                          273008   
          
               Area (sq. mi.) Pop. Density (per sq. mi.) Coastline (coast/area ratio)  \textbackslash{}
          52             9250                       84,8                         7,01   
          58            15007                       70,8                         4,70   
          140             102                       92,5                        39,22   
          223          266000                        1,0                         0,42   
          
              Net migration Infant mortality (per 1000 births)  GDP (\$ per capita)  \textbackslash{}
          52           0,43                               7,18             19200.0   
          58              0                              47,41               500.0   
          140             0                               7,35              3400.0   
          223           NaN                                NaN                 NaN   
          
              Literacy (\%) Phones (per 1000) Arable (\%) Crops (\%) Other (\%) Climate  \textbackslash{}
          52          97,6               NaN       7,79      4,44     87,77       3   
          58          58,6               NaN       4,71      0,67     94,62       2   
          140         97,0               NaN         20         0        80       2   
          223          NaN               NaN       0,02         0     99,98       1   
          
              Birthrate Deathrate Agriculture Industry Service  
          52      12,56      7,68       0,037    0,198   0,765  
          58      26,99      6,24       0,085    0,231   0,684  
          140     17,59       7,1         NaN      NaN     NaN  
          223       NaN       NaN         NaN      NaN     0,4  
\end{Verbatim}
            
    \begin{Verbatim}[commandchars=\\\{\}]
{\color{incolor}In [{\color{incolor}107}]:} \PY{n}{tab}\PY{p}{[}\PY{n}{tab}\PY{p}{[}\PY{l+s+s1}{\PYZsq{}}\PY{l+s+s1}{GDP (\PYZdl{} per capita)}\PY{l+s+s1}{\PYZsq{}}\PY{p}{]}\PY{o}{.}\PY{n}{isnull}\PY{p}{(}\PY{p}{)}\PY{p}{]}
\end{Verbatim}


\begin{Verbatim}[commandchars=\\\{\}]
{\color{outcolor}Out[{\color{outcolor}107}]:}              Country                               Region  Population  \textbackslash{}
          223  Western Sahara   NORTHERN AFRICA                          273008   
          
               Area (sq. mi.) Pop. Density (per sq. mi.) Coastline (coast/area ratio)  \textbackslash{}
          223          266000                        1,0                         0,42   
          
              Net migration Infant mortality (per 1000 births)  GDP (\$ per capita)  \textbackslash{}
          223           NaN                                NaN                 NaN   
          
              Literacy (\%) Phones (per 1000) Arable (\%) Crops (\%) Other (\%) Climate  \textbackslash{}
          223          NaN               NaN       0,02         0     99,98       1   
          
              Birthrate Deathrate Agriculture Industry Service  
          223       NaN       NaN         NaN      NaN     0,4  
\end{Verbatim}
            

    % Add a bibliography block to the postdoc
    
    
    
    \end{document}
